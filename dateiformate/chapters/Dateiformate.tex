%%
%% Copyright (C) 2020 by Manuel Werder
%%
%% This work may be distributed and/or modified under the
%% conditions of the LaTeX Project Public License, either version 1.3
%% of this license or (at your option) any later version.
%% The latest version of this license is in
%%
%%     http://www.latex-project.org/lppl.txt
%%
%! Date = 16.05.20

\secStyle{Dateiformate}
Im Web ist der Einsatz vom korrekten Dateiformat sehr wichtig, dabei gelten als die wichtigsten Formate GIF, JPG, und PNG.
Alle diese Formate haben ihre Vor- und Nachteile die zu beachten sind, wenn es um die Speicherung der Dateien geht.
Es geht im wesentlichen darum die optimale Wiedergabequalität bei einer minimalen Spachergrösse zu erreichen.
GIF, JPG und PNG sind Formate für Pixelgrafiken, wobei jedes Pixel mit Farbe befüllt wird, dabei kann man das Raster
erkennen, wenn man stark hineinzoomt.

\subSecStyle{GIF Dateiformat}
GIF steht für \dq Grafics Interchange Format\dq und bietet einen Farbraum mit 256 Farben, dies kann oftmals zu wenig
sein und daher sollte für detailreiche Bilder nicht dieses Format verwendet werden.
Es eignet sich aber gut für Logos, da diese oft nicht die gleichen Details aufweisen müssen wie ein Hintergrundbild als Beispiel.
GIFs können auch animiert werden und eignen sich für das visuelle Storytelling auf einer Webseite.
Sie besitzen eine relativ kleine Dateigrösse wie auch eine Transparenzstufe.

\subSecStyle{JPG Dateiformat}
Das JPG (oder JPEG) Dateiformat steht für \dq Joint Photographic Experts Group\dq und ermöglicht eine verlustbehaftete
Komprimierung, die als Ergebnis eine kleiner Dateigrösse aufweist.
Wenn aber zu stark komprimiert wird, kann dabei Qualität verloren gehen und es treten unschöne Artefakte auf.
Dieser verlust von Qualität ist schneller Sichtbar bei qualitativ hochwertigen Bildern.
Ein Vorteil von JPG ist die mögliche Vielzahl von Farben, die man darstellen kann.
JPG ist aber nicht mehr das aktuellste Dateiformat und diverse Browserhersteller bevorzugen neue Formate:
\begin{itemize}
    \item JPEG 2000 (Apple Safari)
    \item JPEG XR (Microsoft mit IE 11 und Edge)
    \item WebP (Google Chrome und Opera)
\end{itemize}
Bisher hat sich aber noch keines dieser Dateiformate durchgesetzt, diese bieten jedoch eine höhere visuelle Qualität bei
einer stärkeren Kompressionsrate.

\subSecStyle{PNG Dateiformat}
Das PNG Dateiformat steht für \dq Portable Network Graphics\dq und kommt in zwei Varianten.
PNG-8 und PNG-24 stehen hier zur auswahl.
PNG ermöglicht eine verlustfreie kompression.





