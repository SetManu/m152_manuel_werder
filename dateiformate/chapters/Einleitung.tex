%%
%% Copyright (C) 2020 by Manuel Werder
%%
%% This work may be distributed and/or modified under the
%% conditions of the LaTeX Project Public License, either version 1.3
%% of this license or (at your option) any later version.
%% The latest version of this license is in
%%
%%     http://www.latex-project.org/lppl.txt
%%
%! Date = 13.05.20

\secStyle{Einleitung}
\label{einleitung}
\setcounter{page}{1}
\normalsize
Dieser Teil wurde eins zu eines aus dem Kapitel 6 im Script~\cite{sterchi} entnommen:\newline\newline
Sie kennen viele Programme, welche mit den diversesten Daten arbeiten.
Seien es Text-Dokumente, Filme, Bilder, Grafiken, Musik und Anderes mehr.\newline\newline
Es leuchtet Ihnen sicherlich ein, dass sich eine Textdatei nicht eignet um Videos zu bearbeiten.
Was ist es denn, was den Unterschied macht?
Texte werden in Einheiten zu ein oder zwei Byte (ASCII oder Unicode Zeichen) gespeichert.
Andere Daten wie Musik, Bilder aber auch Daten einer Datenbank, werden als Bitströme auf einem Datenträger abgelegt.
Im Modul 114 habe Sie sich bereits näher mit einigen Datenformaten auseinandergesetzt.\newline\newline
Musik–, Ton, oder Video-Dateien können schnell sehr gross werden.
Darum werden solche Daten, meist verlustbehaftet, komprimiert.
Es existieren mehrere Datenformate für die drei eben erwähnten Medien-Dateien.
Jedes Datenformat hat gewisse Vor– und Nachteile so wie Einsatz-Zwecke.

\subSecStyle{Arbeitsanweisung}
Diese werden gemäss Script wie folgt aufgelistet:
\begin{itemize}
    \item Suchen Sie im Internet nach Information zu Datentypen (Dateitypen), welche es Ihnen, in einem
    zweiten Schritt, gestatten möglichst optimale Datenformate für alle möglichen Medien (Foto,
    Video, Audio, Streaming) in ihrer Website zu wählen
    \item Erstellen Sie eine Tabelle mit den Vor- und Nachteilen vergleichbarer Formate, so wie deren
    Verwendungszweck
    \item Machen Sie sich Gedanken, wie Inhalte formatiert sein müssten, damit diese \dq gestreamt\dq~werden können
    \item Laden Sie das erstellte Dokument zu dem von der Lehrperson genanntem Zeitpunkt in ihr Portfolio
    und geben Sie es zur schnelleren Bewertung auf moodle ab
\end{itemize}

\subSecStyle{Bewertungskriterien}
\begin{itemize}
    \item Abdeckungsgrad
    \item Vollständigkeit
    \item Richtigkeit
    \item Einhaltung des Abgabetermins
    \item Abgabe im verlangten Format.
    Die Arbeit gilt erst als abgegeben, wenn sie im geforderten Format, hier PDF, vorliegt.
\end{itemize}
\newpage
\subSecStyle{Modulidentifikation}
Die Handlungsziele gemäss Modulidentifikation:

\begin{enumerate}
    \item Auftrag/Vorgabe für ein zu erstellendes Multimedia-Element
    analysieren, Voraussetzungen der Einbindung in einen
    Webauftritt untersuchen (Browser, Plattformen, Skriptsprachen, Sicherheitsvorschriften) und geeignetes Tool
    für die Realisierung auswählen.
    \item Web spezifische Multimedia-Inhalte mit geeigneten
    Werkzeugen erstellen.
    \item Prozeduren zur Aufbereitung, Optimierung und Integration
    von Bildern, Grafiken, Animationen, Tondokumenten oder
    Video-Inhalten erstellen und testen.
    \item Integrierte Multimedia-Inhalte auf unterschiedlichen
    stationären und mobilen Plattformen und mit verschiedenen
    Browsern testen, allenfalls adaptieren und freigeben.
\end{enumerate}

\subSecStyle{Projektübersicht}
Es werden hier die wichtigsten Projekttools die wie Beteiligten aufgezeigt.

\begin{table}[h!]
    \centering
    \begin{tabularx}{0.8\textwidth} {
    | >{\raggedright\arraybackslash}X
    | >{\raggedright\arraybackslash}X | }
        \hline
        \sffamily Student und Schreiber & \sffamily  Manuel Werder \\
        \hline
        \sffamily Dozent & \sffamily Daniel Senften  \\
        \hline
    \end{tabularx}
    \caption{\sffamily Projektbeteiligte}
    \label{tab:1}
\end{table}

\begin{table}[h!]
    \centering
    \begin{tabularx}{0.8\textwidth} {
    | >{\raggedright\arraybackslash}X
    | >{\raggedright\arraybackslash}X | }
        \hline
        \LaTeX & \sffamily \href{https://www.overleaf.com}{Overleaf} \\
        \hline
        \sffamily IDE & \sffamily \href{https://www.jetbrains.com/idea/}{IntelliJ IDEA} \\
        \hline
%        \sffamily GitHub & \sffamily \href{https://github.com/SetManu/SpaceWar}{Space War}  \\
%        \hline
%        \sffamily Draw IO & \sffamily \href{https://drawio-app.com/}{Draw IO}  \\
%        \hline
    \end{tabularx}
    \caption{\sffamily Tools, Software und Services}
    \label{tab:2}
\end{table}
