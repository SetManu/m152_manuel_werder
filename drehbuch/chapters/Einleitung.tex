%%
%% Copyright (C) 2020 by Manuel Werder
%%
%% This work may be distributed and/or modified under the
%% conditions of the LaTeX Project Public License, either version 1.3
%% of this license or (at your option) any later version.
%% The latest version of this license is in
%%
%%     http://www.latex-project.org/lppl.txt
%%
%! Date = 13.05.20

\secStyle{Einleitung}
\label{einleitung}
\setcounter{page}{1}
\normalsize
Im Modul 152 soll ein kurz Film erstellt werden, dazu wird ein Drehbuch benötigt, dass hier erarbeitet werden soll.

\subSecStyle{Anweisung}
Diese werden wie folgt aufgelistet:
\begin{itemize}
    \item Ich kann ein Drehbuch für ein Video mit allen notwendigen Einstellungen und Materialien verfassen.
\end{itemize}

\subSecStyle{Auftrag}
Wie bei jeder Filmproduktion empfiehlt es sich, die Herstellung eines Videos genau zu planen.
Dies ist nicht nur aus inhaltlichen Gründen wichtig, sondern gerade bei zeitaufwendigen Trickfilmtechniken notwendig,
um überflüssige Arbeit zu vermeiden.
In diesem \href{https://moodle.talent-factory.ch/login/index.php}{Video} wird dieser Arbeitsschritt einleuchtend beschrieben:
\begin{itemize}
    \item Schreiben Sie einen ersten Entwurf für Ihr Drehbuch
    \item Benutzen Sie dazu die "Checkliste" sowie die "Drehbuchvorlage".
\end{itemize}

\subSecStyle{Checkliste}
Checkliste, Vorbereitung für das Drehbuch (\cite{talent}):
\begin{enumerate}
    \item Grundlegendes
    \begin{itemize}
        \item Inhalt und Reihenfolge festlegen (deduktiv / induktiv)
        \item Story für den Screencast entwickeln- keine reine Aufzählung von Features, sondern eine Dramaturgie entwickeln
    \end{itemize}
    \item Regieanweisungen
    \begin{itemize}
        \item In Textform, Skizzen und/oder Screenshots festhalten, welche Bildelemente im Screencast gezeigt werden
        \item Einsatz von weiteren Text- und Strukturelementen
        \item Einsatz von Markierungen und Hervorhebungen
        \item Gegebenenfalls Beispieltexte für Textfelder festlegen
    \end{itemize}
    \item Sprechtext
    \begin{itemize}
        \item Gesprochene Inhalte für die jeweiligen Bildschirmaktionen formulieren
    \end{itemize}
    \item Dauer
    \begin{itemize}
        \item Gesamtdauer in etwa festlegen
        \item Ungefähre Dauer für gesprochene Texte einzelner Szenen festlegen
    \end{itemize}
    \item Intro und Abspann
    \begin{itemize}
        \item Intro Folie erstellen - Titel und Einleitung, was kann der Nutzer erwarten?
        \item Abspann Folie erstellen - Resümee des Videos, Folgeaktion anbieten
    \end{itemize}
    \item Gegenlesen
    \begin{itemize}
        \item Drehbuch jemandem zum Gegenlesen geben
    \end{itemize}
\end{enumerate}
\newpage

\subSecStyle{Projektübersicht}
Es werden hier die wichtigsten Projekttools die wie Beteiligten aufgezeigt.
\begin{table}[h!]
    \centering
    \begin{tabularx}{0.8\textwidth} {
        | >{\raggedright\arraybackslash}X
        | >{\raggedright\arraybackslash}X | }
        \hline
        \sffamily Student und Schreiber & \sffamily  Manuel Werder \\
        \hline
        \sffamily Dozent & \sffamily Daniel Senften  \\
        \hline
    \end{tabularx}
    \caption{\sffamily Projektbeteiligte}
    \label{tab:1}
\end{table}
\begin{table}[h!]
    \centering
    \begin{tabularx}{0.8\textwidth} {
        | >{\raggedright\arraybackslash}X
        | >{\raggedright\arraybackslash}X | }
        \hline
        \LaTeX & \sffamily \href{https://www.overleaf.com}{Overleaf} \\
        \hline
        \sffamily IDE & \sffamily \href{https://www.jetbrains.com/de-de/idea/}{IntelliJ IDEA}  \\
        \hline
        \sffamily Draw IO & \sffamily \href{https://drawio-app.com/}{Draw IO}  \\
        \hline
    \end{tabularx}
    \caption{\sffamily Tools, Software und Services}
    \label{tab:2}
\end{table}

