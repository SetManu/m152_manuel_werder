%%
%% Copyright (C) 2020 by Manuel Werder
%%
%% This work may be distributed and/or modified under the
%% conditions of the LaTeX Project Public License, either version 1.3
%% of this license or (at your option) any later version.
%% The latest version of this license is in
%%
%%     http://www.latex-project.org/lppl.txt
%%
%! Date = 13.05.20

\secStyle{Einleitung}
\label{einleitung}
\setcounter{page}{1}
\normalsize
Es soll gemäss dem Auftrag aus Kapitel 5 im Script~\cite{sterchi} folgender Inhalt geklärt werden:\newline\newline
Medien sind überall auf dem Internet zu finden.
Suchmaschinen finden zu sehr vielen Begriffen,
Text–, Bild–, Ton oder Film Dokumente.
Doch dürfen Sie diese Medien einfach so, zum Beispiel in
eigenen Websites, verwenden?\newline\newline
Wenn die Frage so gestellt wird, wohl eher nicht.
Aber wie ist die Wiederverwendung genau geregelt?\newline\newline
Mal angenommen Sie befinden sich an einem öffentlichen Ort.
Sie fotografieren an ihnen vorbeischlendernde Personen.
Mütter mit ihren Kindern, Geschäftsleute im Gespräch vertieft, Liebespaare, und, und, und.
Dürfen Sie die Bilder, die Fotos nun auf ihrer Website veröffentlichen?
Schliesslich habe Sie ja die Fotos gemacht?\newline\newline
Und wieder, wenn die Frage so gestellt wird, wohl eher nicht.
Was muss man tun, damit eine Veröffentlichung gestattet ist?\newline\newline
Wie verhält es sich, wenn Sie Personen an einer Demonstration fotografieren?

\subSecStyle{Arbeitsanweisung}
Diese werden gemäss Script wie folgt aufgelistet:
\begin{itemize}
    \item Suchen Sie im Internet nach Informationen welche die obigen Fragen beantworten.
    Begriffe, welche im Dokument vorhanden sein müssen sind unter Anderem: \dq Urheberrecht", "das Recht am eigenen Bild", \dq Copyright".
    Der Umfang des Dokumentes ist mindestens zwei DIN - A4 Seiten.
    \item Wie ist das mit dem Urheberrecht in anderen Ländern geregelt, zum Beispiel in Deutschland,
    in den USA? Interessiert uns das im Zusammenhang mit einer Website überhaupt?
    \item Laden Sie das erstellte Dokument zu dem von der Lehrperson genanntem Zeitpunkt in ihr
    Portfolio und geben Sie es zur schnelleren Bewertung auf moodle ab.
\end{itemize}

\subSecStyle{Bewertungskriterien}
\begin{itemize}
    \item Abdeckungsgrad
    \item Vollständigkeit
    \item Richtigkeit
    \item Einhaltung des Abgabetermins
    \item Abgabe im verlangten Format.
    Die Arbeit gilt erst als abgegeben, wenn sie im geforderten Format, hier PDF, vorliegt.
\end{itemize}
\newpage
\subSecStyle{Modulidentifikation}
Die Handlungsziele gemäss Modulidentifikation:

\begin{enumerate}
    \item Auftrag/Vorgabe für ein zu erstellendes Multimedia-Element
    analysieren, Voraussetzungen der Einbindung in einen
    Webauftritt untersuchen (Browser, Plattformen, Skriptsprachen, Sicherheitsvorschriften) und geeignetes Tool
    für die Realisierung auswählen.
    \item Web spezifische Multimedia-Inhalte mit geeigneten
    Werkzeugen erstellen.
    \item Prozeduren zur Aufbereitung, Optimierung und Integration
    von Bildern, Grafiken, Animationen, Tondokumenten oder
    Video-Inhalten erstellen und testen.
    \item Integrierte Multimedia-Inhalte auf unterschiedlichen
    stationären und mobilen Plattformen und mit verschiedenen
    Browsern testen, allenfalls adaptieren und freigeben.
\end{enumerate}

\subSecStyle{Projektübersicht}
Es werden hier die wichtigsten Projekttools die wie Beteiligten aufgezeigt.

\begin{table}[h!]
    \centering
    \begin{tabularx}{0.8\textwidth} {
    | >{\raggedright\arraybackslash}X
    | >{\raggedright\arraybackslash}X | }
        \hline
        \sffamily Student und Schreiber & \sffamily  Manuel Werder \\
        \hline
        \sffamily Dozent & \sffamily Daniel Senften  \\
        \hline
    \end{tabularx}
    \caption{\sffamily Projektbeteiligte}
    \label{tab:1}
\end{table}

\begin{table}[h!]
    \centering
    \begin{tabularx}{0.8\textwidth} {
    | >{\raggedright\arraybackslash}X
    | >{\raggedright\arraybackslash}X | }
        \hline
        \LaTeX & \sffamily \href{https://www.overleaf.com}{Overleaf} \\
        \hline
        \sffamily IDE & \sffamily \href{https://www.jetbrains.com/idea/}{IntelliJ IDEA} \\
        \hline
%        \sffamily GitHub & \sffamily \href{https://github.com/SetManu/SpaceWar}{Space War}  \\
%        \hline
%        \sffamily Draw IO & \sffamily \href{https://drawio-app.com/}{Draw IO}  \\
%        \hline
    \end{tabularx}
    \caption{\sffamily Tools, Software und Services}
    \label{tab:2}
\end{table}
