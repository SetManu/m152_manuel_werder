%%
%% Copyright (C) 2020 by Manuel Werder
%%
%% This work may be distributed and/or modified under the
%% conditions of the LaTeX Project Public License, either version 1.3
%% of this license or (at your option) any later version.
%% The latest version of this license is in
%%
%%     http://www.latex-project.org/lppl.txt
%%
%! Date = 16.05.20

\secStyle{Das Werk}
\label{das-werk}
Bevor die Fragestellungen aus dem Kapitel~\ref{einleitung} beantwortet werden, sollen diverse Begriffe erläutert werden.
Das Augenmerk fällt dabei auf die zwei folgenden Bereiche.
\begin{itemize}
    \item Urheberrecht (Copyright auf Englisch)
    \item Das Recht am eigenen Bilde
\end{itemize}

\subSecStyle{Das Urheberrecht}
Im Urheberrechtsgesetz ist festgelegt wie der Urheber eines Werkes der Literatur und Kunst geschützt ist, dabei wird die
Art und Weise wie eine Idee zu Ausdruck gebracht wird geschützt, aber nicht etwa das grundlegende Konzept der Idee selbst.
Der Schutz des Urhebers bezieht sich also auf die Form des Werks und nicht auf dessen Inhalt.
Das schweizerische Gesetz darüber kann man hier nachlesen,
\href{https://www.admin.ch/opc/de/classified-compilation/19920251/index.html}{Bundesgesetz über das Urheberrecht und verwandte Schutzrechte}.
\newpage
\subSubSecStyle{Was gilt nun als "Werk"?}
\begin{itemize}
    \item Jedes literarische Werk, also Texte in jedweder Form.
    Der Quelltext von Software gehört auch dazu, davon ist aber der eigentliche Algorithmus ausgeschlossen,
    der der jeweiligen Software zugrunde liegt.
    \item Fotografie und Filme
    \item Akustische Werke wie Musik.
    \item Bildende Kunst (Malerei, Bildhauerei, Grafik, etc.)
    \item Gegenstände mit Gebrauchswert
    \item Wissenschaftliche oder technische Werke wie etwa Zeichnungen, Pläne, Karten, Diagramme oder plastische Darstellungen.
    \item Architektur, also Baukunst
    \item Performance und Pantomime
\end{itemize}

\subSubSecStyle{Keinen Schutz für}
Ideen, Leistungen, Konzepte oder Anweisungen werden nicht durch das Urheberrecht geschützt, dies gilt auch dann nicht,
wenn diese individuell sind.
Auch nicht geschützt sind Gesetze, Verordnungen und amtliche Erlasse wie Entscheidungen,
Protokolle und Berichte von Behörden, die die Rechtsstellung des Bürgers betreffen.

\subSubSecStyle{Dauer des Schutzes}
Im Allgemeinen erlischt der Urheberrechtsschutz nach 70 Jahren nach dem Tod des Urhebers (50 Jahre bei Software).
Bei der Fotografie nach 50 Jahren der Herstellung.

\subSubSecStyle{Ab wann gilt das Urheberrecht?}
Der Urheberrechtsschutz entsteht automatisch bei der Schöpfung des Werks, dabei bedarf es keinerlei Formalitäten oder einer Anmeldung.
In der Schweiz werden darüber keine Register geführt.

\subSubSecStyle{International}
Es gilt Grundsätzlich immer das Rechtssystem der jeweiligen Nation, somit schützt das Urheberrecht die Urheber und ihre
Werke nur in der Schweiz, damit aber auch auf internationaler Ebene ein Schutz aufgebaut werden kann, wird dies durch
\href{https://www.ige.ch/de/recht-und-politik/immaterialgueterrecht-national/urheberrecht.html}{internationale Abkommen} geregelt.

\subSubSecStyle{Verwendung eines Werks}
Wenn man nun ein fremdes Werk auf einer Website verwenden will oder auch allgemein im öffentlichen Bereich, so muss
man die Erlaubnis des Urhebers einholen.
Eine solche Erlaubnis kommt in Form einer Lizenz, in dieser ist auch definiert,
auf welche Art und Weise ein Werk verwendet werden darf.
Es kann nun aber sein, dass es im öffentlichen Interesse ist ein bestimmtes Werk ohne weitere Lizenz zu verwenden und
diese Nutzung höher eingeschätzt wird als die Interesse des Urhebers, ein solcher Fall wäre da, wenn ein spezifisches
Werk für den schulischen Zweck verwendet wird.

\subSubSecStyle{Public Domain}
Der Urheberrechtsschutz läuft nach einer bestimmten Zeit ab, wenn dies geschehen ist, darf man das Werk nach seinem Belieben verwenden.
Es gilt dan ab diesem Moment als Allgemeingut, dies ist auch der Grund, wieso man jetzt einfach so ein Cello nehmen
und im öffentlichen bereich Bach Cello Suite No. 1 vorspielen kann.
Was man dann wiederum nicht kann, ist die Aufnahme auf der CD von einem Maestro,
zum Beispiel einfach so vervielfältigen und im Internet verkaufen, da hier wieder ein Urheberrecht auf die erstellte CD gilt.
Zudem darf man nicht mehr geschützte Werke nach seinem belieben verändern.
Das soll aber nicht heissen das man jetzt das Sacheigentum eines Anderen beschädigen darf.

\subSubSecStyle{Creative Commons}
Diverse Urheber von Werken stellen diese untere eine freie Lizenz, wie etwa Open Licenses, Creative Commons License, MIT License
oder GNU-Lizenz, dabei ist aber wichtig das auch dort die minimalen Bedingungen der Lizenz beachtet werden, diese
ist meist so, das der Urheber des Werks namentlich erwähnt sein soll.
Falls so eine Vertragsverletzung vorliegt, könnte es zu etwaigen Kosten kommen.
Es ist auch wichtig sicherzustellen, dass der Lizenzgeber auch tatsächlich über die Rechte verfügt, eine solche freie
Lizenz zu vergeben.
Es kann sein, das dieser seine Rechte am jeweiligen Werk bereits früher abgetreten hat, uns somit der Lizenznehmer auf
Schadensersatz verklagt werden kann, da hier der Grundsatz des guten Glaubens nicht schützt.


%\subSecStyle{Internationales Urheberrecht}



\subSecStyle{Das Recht am eigenen Bilde}
Unabhängig von der urheberrechtlichen Sicht, besteht immer das Recht am eigenen Bild.
Was bedeutet, dass die jeweiligen Personen, die abgelichtet wurden, darüber entscheiden,
ob nun und auch in welcher Form das Bild veröffentlicht werden darf.
Deswegen dürfen auch nur Fotos dann veröffentlicht werden, wenn die Abgelichteten ihr Einverständnis geben.
\newline\newline
Es kann unter bestimmten Umständen auf eine Einwilligung verzichtet werden, wenn dabei ein überwiegendes
öffentliches oder privates Interesse besteht und so eine Veröffentlichung rechtfertigt.
Dies sollte aber mit Vorsicht getätigt werden, besonders in dem Fall, wenn die Bilder einzelner Personen zur
Veröffentlichung stehen.
Bei Zweifel sollte stets die Einwilligung der Betroffenen Personen eingeholt werden.
Unabhängig davon, ob es sich nun um aktuelle Bilder oder Fotos, die vor Jahren aufgenommen wurden, gilt es immer noch,
das Persönlichkeitsrecht der jeweiligen Personen zu waren.
Dieses Recht besteht so lange wie diese Personen leben und es kann jederzeit von den Personen geltend gemacht werden.

\subSubSecStyle{Fotos von Gruppen}
\label{gruppenfotos}
Bei Fotos von Gruppen kann es zu einem Eingriff in das Persönlichkeitsrecht kommen, wenn einzelne Personen aus der Gruppe hervorstechen.
Wie und wann dies zutrifft muss nach den konkreten Einzelfällen beurteilt werden.
Die widerrechtliche Veröffentlichung kann nur dann ausgeschlossen werden, wenn keine der jeweiligen Personen identifizierbar ist.
Im Falle von Zweifel sollte vor einer Veröffentlichung die Einwilligung der identifizierbaren Personen eingeholt werden.

\subSubSecStyle{Rechtsgültige Einwilligung}
Eine rechtsgültige Einwilligung ist nur dann gültig, wenn eine betroffene Person angemessen über den Nutzen der Fotografie
informiert wird und freiwillig einer Veröffentlichung zustimmt.
Darüber zu informieren in welcher weise die Bilder verwendet werden, heist zu erklären in welcher Form die jeweiligen Fotos
veröffentlicht werden (Internet, Printmedien, Werbeflyer, etc.).

\subSubSecStyle{Rückzug der Einwilligung}
Grundsätzlich kann eine erteilte Einwilligung jederzeit wieder zurückgezogen werden.
Was zur Folge hat, dass eine Veröffentlichung, insofern möglich, rückgängig gemacht werden muss.
Wenn ein solcher Rückzug Schaden verursacht, kann die zurückziehende Person verpflichtet werden einen Teil
des Schadens zu übernehmen.

\subSubSecStyle{Veröffentlichung ohne Rechtfertigungsgrund}
Personen, von denen Bilder ohne Rechtfertigung veröffentlicht wurden, können sich jederzeit gegen diese Veröffentlichung
wehren und ihre Ansprüche mittels Zivilklage einfordern.
Wenn das Gericht zum Schluss kommt, dass es eine widerrechtliche Persönlichkeitsverletzung gibt, kann neben der
Entfernung der Bilder auch Schadensersatz und/oder eine Genugtuung angeordnet werden.
Es muss zudem damit gerechnet werden, dass die Gerichtskosten für den Kläger zu übernehmen sind.


\secStyle{Fragestellungen erläutern}
Es werden in diesem Kapitel die Fragestellungen aus Kapitel~\ref{einleitung} beantwortet.
Die Fragestellungen werden einfach halber direkt am Anfang vom Unterkapitel aufgeführt, um so auch immer
direkt darauf bezug nehmen zu können.

\subSecStyle{Erste Fragestellung}
Medien sind überall auf dem Internet zu finden.
Suchmaschinen finden zu sehr vielen Begriffen,
Text–, Bild–, Ton oder Film Dokumente.
Doch dürfen Sie diese Medien einfach so, zum Beispiel in
eigenen Websites, verwenden?\newline\newline
Wenn die Frage so gestellt wird, wohl eher nicht.
Aber wie ist die Wiederverwendung genau geregelt?\newline\newline
Grundsätzlich gilt, falls nicht anders nachgewiesen, für alle Medien in jedweder Form im Internet, das diese
durch das Urheberrecht geschützt sind, dadurch muss zuerst abgeklärt werden in welcher Form das eventuelle
Werk geschützt ist und dem entsprechend auf einem rechtsgültigen Weg sich eine Nutzungslizenz zu beschaffen.


\subSecStyle{Zweite Fragestellung}
Mal angenommen Sie befinden sich an einem öffentlichen Ort.
Sie fotografieren an ihnen vorbeischlendernde Personen.
Mütter mit ihren Kindern, Geschäftsleute im Gespräch vertieft, Liebespaare, und, und, und.
Dürfen Sie die Bilder, die Fotos nun auf ihrer Website veröffentlichen?
Schliesslich habe Sie ja die Fotos gemacht?\newline\newline
Und wieder, wenn die Frage so gestellt wird, wohl eher nicht.
Was muss man tun, damit eine Veröffentlichung gestattet ist?\newline\newline
Jede natürliche Person hat das Recht am eigenen Bilde, somit ist klar, dass die Fotos ohne die Zustimmung der jeweiligen
Person nicht verwendet werden dürfen, die den persönlichen Nutzungsbereich überschreiten.
Es gibt da aber gewisse Ausnahmen, wenn keine spezifische Person direkt zu identifizieren ist, so gilt dies nicht.
Falls nun aber doch einzelne Personen zu identifizieren sind, so gilt es diese zu informieren über den
Verwendungszweck der Fotografie und deren Einwilligung einzuholen für die Verwendung.

\subSecStyle{Dritte Fragestellung}
Wie verhält es sich, wenn Sie Personen an einer Demonstration fotografieren?\newline\newline
Auch hier gelten die gleichen Weisungen wie aus der \dq Zweiten Fragestellung\dq, dabei muss man aber gewisse Tatsachen noch
genauer betrachten.
Wenn man nun eine Straftat fotografiert hat, so muss dieses Foto an eine entsprechende Behörde weiter gegeben werden.
Da es eine Demonstration ist, kann es gut sein das diverse Personen vermummt sind, somit tragen die jeweiligen Personen
selbst dazu bei, dass ihre identität geschützt ist.
Es ist auch so, dass dies einer jener Fälle ist, wo das öffentliche Interesse das des privaten
überschreitet und so unter gegebenen Umständen eine Veröffentlichung rechtfertigt.



































