%%
%% Copyright (C) 2020 by Manuel Werder
%%
%% This work may be distributed and/or modified under the
%% conditions of the LaTeX Project Public License, either version 1.3
%% of this license or (at your option) any later version.
%% The latest version of this license is in
%%
%%     http://www.latex-project.org/lppl.txt
%%
%! Date = 13.05.20

\secStyle{Einleitung}
\label{einleitung}
\setcounter{page}{1}
\normalsize
some text...

\subSecStyle{Vorgaben}
Mindestanforderungen an die Website gemäss~\cite{sterchi}:
\begin{itemize}
    \item Die Site verfügt über eine ansprechendes, zum Projekt passendes, Design
    \item Für den Bau der Site wird HTML5 verwendet (bekannt aus Modul 101)
    \item Die im Verlauf des Semesters erstellten Medien werden mit Hilfe von geeigneten Mitteln in die
    Site eingebunden. Erlaubt sind Plug-ins, JavaScript und oder PHP, so wie HTML-Tags. Zum Zeit-
    punkt, an dem Sie die Site fertig stellen müssen, werden Sie über genügend «Know-How» in
    JavaScript und PHP verfügen. PHP ist kein «muss» für dieses Projekt
    \item Für die Website darf kein bereits existierendes Foto-Galerie Plug-in, oder ähnliches, verwendet werden
    \item Die Site muss über ein «responsive Design» verfügen, will heissen, das Aussehen der Webseite
    passt sich den Möglichkeiten des Anzeigegerätes an (Browser auf einem Desktop-Gerät, Browser
    auf einem Tablett, Browser auf einem Handy)
    \item Die Website muss mit mindestens drei Browsern getestet und für gut befunden worden sein
    \item Auf der Site muss ein, von einem öffentlichen Server, nicht von ihnen erstellter, Videostream
    \item integriert werden
\end{itemize}
Was darf zur Erstellung der Site verwendet, beziehungsweise, nicht verwendet werden?
\begin{itemize}
    \item Es dürfen keine CMS, wie typo3, Joomla oder Wordpress eingesetzt werden
    \item Sie dürfen Datenbanken verwenden
    \item Sie dürfen JavaScript verwenden
    \item Sie dürfen PHP verwenden
    \item Sie dürfen AJAX verwenden
\end{itemize}

\subSecStyle{Modulidentifikation}
Die Handlungsziele gemäss Modulidentifikation:

\begin{enumerate}
    \item Auftrag/Vorgabe für ein zu erstellendes Multimedia-Element analysieren, Voraussetzungen der Einbindung in
    einen Webauftritt untersuchen (Browser, Plattformen, Scriptsprachen, Sicherheitsvorschriften) und geeignetes Tool für die Realisierung auswählen.
    \begin{enumerate}
        \item Kennt Möglichkeiten zur Evaluation eingesetzter Software bei einem Webauftritt und kann das Zusammenspiel
        der Software (Tools und Script-/Programmiersprachen, usw.) an Beispielen darlegen.
        \item Kennt die rechtlichen Bestimmungen des Urheberrechts und kann darlegen, welche Aspekte bei der Verwendung
        fremder Komponenten in einem Web-Auftritt zu beachten sind.
        \item Kennt gängige Dateiformate für Multimedia-Inhalte und kann aufzeigen für welche Art von Inhalten bzw.
        welche Art von Anwendungen sich diese eignen.
        \item Kennt Grundfunktionen von Tools, mit denen Multimedia-Elemente generiert und bearbeitet werden können.
    \end{enumerate}
    \item Webspezifische Multimedia-Inhalte mit geeigneten Werkzeugen erstellen.
    \begin{enumerate}
        \item Kennt Techniken und Werkzeuge (Tools) zur Aufbereitung, Adaptierung und Optimierung von webspezifischen Multimedia-Inhalten.
    \end{enumerate}
    \item Prozeduren zur Aufbereitung, Optimierung und Integration von Bildern, Grafiken, Animationen, Tondokumenten
    oder Video-Inhalten erstellen und testen.
    \begin{enumerate}
        \item Kennt Techniken zur Integration von Multimedia-Inhalten (Streaming, externe Inhalte, Scripts).
        \item Kennt Befehle für die Automatisierung von Prozeduren, welche der automatischen Konvertierung und
        Integration von Multimedia-Inhalten dienen.
    \end{enumerate}
    \item  	Integrierte Multimedia-Inhalte auf unterschiedlichen stationären und mobilen Plattformen und mit verschiedenen
    Browsern testen, allenfalls adaptieren und freigeben.
    \begin{enumerate}
        \item Kennt Möglichkeiten die Eigenschaften von Browsern sowie deren Unterstützung von Multimedia-Inhalten zu eruieren.
        \item Kennt das Plug-In Konzept der Browser und kann dementsprechend Komponenten auswählen oder erstellen.
        \item Kennt Möglichkeiten, Multimedia-Inhalte auf stationären und mobilen Plattformen abzubilden.
    \end{enumerate}
\end{enumerate}

\subSecStyle{Projektübersicht}
Es werden hier die wichtigsten Projekttools die wie Beteiligten aufgezeigt.

\begin{table}[h!]
    \centering
    \begin{tabularx}{0.8\textwidth} {
    | >{\raggedright\arraybackslash}X
    | >{\raggedright\arraybackslash}X | }
        \hline
        \sffamily Student und Projektleiter & \sffamily  Manuel Werder \\
        \hline
        \sffamily Dozent & \sffamily Daniel Senften  \\
        \hline
    \end{tabularx}
    \caption{\sffamily Projektbeteiligte}
    \label{tab:1}
\end{table}

\begin{table}[h!]
    \centering
    \begin{tabularx}{0.8\textwidth} {
        | >{\raggedright\arraybackslash}X
        | >{\raggedright\arraybackslash}X | }
        \hline
        \LaTeX & \sffamily \href{https://www.overleaf.com}{Overleaf} \\
        \hline
        \sffamily GitHub & \sffamily \href{https://github.com/SetManu/m152_manuel_werder}{Repo} \\
        \hline
        \sffamily Heroku & \sffamily \href{https://m152-manuel-werder.herokuapp.com}{Website} \\
        \hline
    \end{tabularx}
    \caption{\sffamily Tools, Software und Services}
    \label{tab:2}
\end{table}
